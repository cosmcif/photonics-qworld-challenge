\documentclass[hidelinks, a4paper,12pt]{article}
\usepackage[utf8]{inputenc}
\usepackage[margin=3cm]{geometry}
\usepackage{url}
\usepackage{hyperref}
\usepackage{dirtree}
\usepackage{xcolor}

\title{Womanium Quantum Hackathon 2022\\Quantum Hardware Education Challenge\\Inception Document}
\author{Temitope Adeniyi, Klára Churá, Deeksha Dadhich,\\Sofia d'Atri, Seyedeh Mahshad Hosseini}
\date{}

\begin{document}
\maketitle

\tableofcontents

\newpage
\section{Introduction}
This document will provide some information about our challenge for the \textbf{Womanium Quantum Hackathon 2022}. It contains the team's details, information regarding the challenge, an outline of our solution and the main features we chose to implement.

\section{Team}
The team name is \textbf{QGirls}. The team is composed of 5 members. The names and the contact details of the members are listed below.\\

\begin{tabular}{p{.275\linewidth} p{.65\linewidth}}
\textbf{Temitope Adeniyi} &\textbf{Discord} Herostar\#3246\\[.3em]
                        &\textbf{GitHub} \href{https://github.com/Temistar}{Temistar}\\[.3em]
                        &\textbf{e-mail} \href{mailto:odeyomitemitope@gmail.com}{odeyomitemitope@gmail.com}\\[.5em]
\\
\textbf{Klára Churá} &\textbf{Discord} clarech712\#4866\\[.3em]
                        &\textbf{GitHub} \href{https://github.com/clarech712}{clarech712}\\[.3em]
                        &\textbf{e-mail} \href{mailto:clarech712@gmail.com}{clarech712@gmail.com}\\[.5em]
\\
\textbf{Deeksha Dadhich} &\textbf{Discord} Deeksha\#8552\\[.3em]
                        &\textbf{GitHub} \href{https://github.com/newbeaen}{newbeaen}\\[.3em]
                        &\textbf{e-mail} \href{mailto:deeksha.dadhich@icfo.eu}{deeksha.dadhich@icfo.eu}\\[.5em]
\\
\textbf{Sofia d'Atri} &\textbf{Discord} cosmcif\#1672\\[.3em]
                        &\textbf{GitHub} \href{https://github.com/cosmcif}{cosmcif}\\[.3em]
                        &\textbf{e-mail} \href{mailto:datrisof@gmail.com}{datrisof@gmail.com}\\[.5em]
\\
\textbf{Seyedeh Mahshad} &\textbf{Discord} Mahshad Hosseini\#3526\\[.3em]
\textbf{Hosseini} &\textbf{GitHub} \href{https://github.com/MahshadHosseini}{MahshadHosseini}\\[.3em]
                        &\textbf{e-mail} \href{mailto:s.mahshadhosseini@gmail.com}{s.mahshadhosseini@gmail.com}\\[.5em]
\end{tabular}

\section{Challenge}
The challenge we chose is the \textbf{Quantum Hardware Education Challenge} by \textbf{QWorld}. In particular, we decided to develop a solution on the topic of Photonics Quantum Computers and we will focus on the creation of a course named \textbf{Introduction to Photonics Quantum Computers}.\\In the following pages, we will discuss our ideas and plans for this challenge.

\newpage
\section{Our challenge solution}

\subsection*{Introduction to Photonics Quantum Computers}
\subsection{Course overview}
Our course is aimed at students with high school knowledge of physics.\\The course will employ different teaching methods to create an engaging learning experience for both in-person and online workshops. Such methods can include, but are not limited to: written material, video lectures, guided tutorials, notebook exercises and quizzes.
\subsection{Modules}
The course will include modules covering different topics. We decided to start with creating 4 core modules. This is a first draft those modules (it may be subject to changes):
\begin{enumerate}
  \item Introduction and motivation
  \item Theory (basic concepts)
  \item Hands on (basic concepts)
  \item Applications (overview)
\end{enumerate}
We plan on adding more modules if the time allows us to. Some examples of modules that can be added are:
\begin{enumerate}
  \item Theory (advanced concepts)
  \item Hands on (advanced concepts)
  \item Specific modules for specific applications
  \item Industry insights
\end{enumerate}
\subsection{Exercises}
Each module will be followed by a set of exercises to test the student's understanding.\\ Theoretical modules will be accompained by a multiple-choices quiz, while hands-on modules will include notebook exercises for students to practice on.

\subsection{Presentation}
We plan on gathering all of our materials on a website, that can be reached \href{https://cosmcif.github.io/photonics-qworld-challenge/index.html}{\underline{here}}.
\end{document}
