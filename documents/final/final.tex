\documentclass[hidelinks, a4paper,12pt]{article}
\usepackage[utf8]{inputenc}
\usepackage[margin=3cm]{geometry}
\usepackage{url}
\usepackage{hyperref}
\usepackage{dirtree}
\usepackage{xcolor}

\title{Womanium Quantum Hackathon 2022\\Quantum Hardware Education Challenge\\Final Submission Document}
\author{Temitope Adeniyi, Klára Churá, Deeksha Dadhich,\\Sofia d'Atri, Seyedeh Mahshad Hosseini}
\date{}

\begin{document}
\maketitle

\tableofcontents

\section{Introduction}
This document will provide some information about our challenge for the \textbf{Womanium Quantum Hackathon 2022}. It contains a summary of all the milestones we hit during the challenge, as well as the changes we made to our original plan. This document is an overview of the final state of the challenge submission.
\newpage
\section{Our goals}
The team originally created a plan for developing the challenge solution that can be found in the Inception document and that was updated in the Midway document. Here is a summary of the goals that we set and reached.

\subsection{Course overview}
Our course is aimed at students with high school knowledge of physics.\\The course will employ different teaching methods to create an engaging learning experience for both in-person and online workshops. Such methods include: textual guides, quizzes, interactive activities and audio lectures.

\subsection{Modules}
The initial plan was to include 4 modules for this course. We then decided to change the structure to create 5+1 modules. The final course curriculum is as follows:
\begin{enumerate}
  \item Introduction and motivation
  \item Photonic Qubits
  \item Photonic Qubit Processing
  \item Photons Circuits
  \item Photonics in Industry and Academia
  \item Extra
\end{enumerate}
The ``Extra'' module contains content that complements the course. Some of this extra content includes a minigame on Quantum Key Distribution. The rest are links to external sources.
\subsection{Exercises}
The time constraint limited the amount of content we could produce, and this is reflected in the exercises as well. One exercise is currently available in Module 3. We plan on expanding the course with more exercises even after the end of this hackathon.
\subsection{Presentation}
We created a website to collect all the materials, it can be reached \href{https://cosmcif.github.io/photonics-qworld-challenge/index.html}{\underline{here}}.\\After the hackathon ends, the website will be kept active and will be expanded with new content.
\newpage
\section{Final milestone}
What has our group achieved?
\begin{enumerate}
  \item The website is ready and working.
  \item A minigame on Quantum Key Distribution has been developed and published.
  \item The core modules have been polished and added to the website. 
  \item Exercises have been created.
  \item Some professionals have been contacted in order to be interviewed. 
  \item We have interviewed a professional in the field. More interviews will follow.
\end{enumerate}

\section{After the hackathon}
The time constraint limits what we can create. Therefore, we believe this project shouldn't finish with the hackathon. We are committed towards mantaining the website and creating more content after the end of the hackathon. \\
Examples of additions that can be made after the end of the hackathon:
\begin{enumerate}
  \item In-depth modules about applications of photonics.
  \item In-depth modules about hardware and materials.
  \item More exercises.
  \item More interactive content such as videogames or video tutorials.
  \item Contributions of experts in the field.
  \item Updated modules about state-of-the-art research and engineering developments.
  \item The creation of a community.
\end{enumerate}
\end{document}

