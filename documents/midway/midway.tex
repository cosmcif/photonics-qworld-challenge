\documentclass[hidelinks, a4paper,12pt]{article}
\usepackage[utf8]{inputenc}
\usepackage[margin=3cm]{geometry}
\usepackage{url}
\usepackage{hyperref}
\usepackage{dirtree}
\usepackage{xcolor}

\title{Womanium Quantum Hackathon 2022\\Quantum Hardware Education Challenge}
\author{Temitope Adeniyi, Klára Churá, Deeksha Dadhich,\\Sofia d'Atri, Seyedeh Mahshad Hosseini}
\date{}

\begin{document}
\maketitle

\tableofcontents


\section{Introduction}
This document will provide some information about our challenge for the \textbf{Womanium Quantum Hackathon 2022}. It contains a summary of all the milestones we hit in the first half of the challenge, as well as the changes we made to our original plan.
\newpage
\section{Our planned goals}
The team originally created a plan for developing the challenge solution that can be found in the Inception document. Here, we'll attach the original goals.

\subsection{Course overview}
Our course is aimed at students with high school knowledge of physics.\\The course will employ different teaching methods to create an engaging learning experience for both in-person and online workshops. Such methods can include, but are not limited to: written material, video lectures, guided tutorials, notebook exercises and quizzes.
\subsection{Modules}
The course will include modules covering different topics. We decided to start with creating 4 core modules. This is a first draft those modules (it may be subject to changes):
\begin{enumerate}
  \item Introduction and motivation
  \item Theory (basic concepts)
  \item Hands on (basic concepts)
  \item Applications (overview)
\end{enumerate}
We plan on adding more modules if the time allows us to. Some examples of modules that can be added are:
\begin{enumerate}
  \item Theory (advanced concepts)
  \item Hands on (advanced concepts)
  \item Specific modules for specific applications
  \item Industry insights
\end{enumerate}
\subsection{Exercises}
Each module will be followed by a set of exercises to test the student's understanding.\\ Theoretical modules will be accompained by a multiple-choices quiz, while hands-on modules will include notebook exercises for students to practice on.

\subsection{Presentation}
We plan on gathering all of our materials on a website, that can be reached \href{https://cosmcif.github.io/photonics-qworld-challenge/index.html}{\underline{here}}.
\newpage
\section{Our new goals}
During the first week, as we worked on the modules and structure of the course, we decided to change some things. Here are the major changes.
\subsection{Course overview}
Our course is a gentle introduction to the structure of Photonics Quantum Computers. The course is divided in modules, each with a written guide and exercises. The course will be hosted on a website, to make it interactive and appealing.
\subsection{Modules}
The initial plan was to include 4 modules for this course. We then decided to change the structure to create 5+1 modules:
\begin{enumerate}
  \item Introduction and motivation
  \item Photonic Qubits
  \item Photonic Circuits
  \item Photons Detection
  \item Photonics in Industry and Accademia
  \item Extra
\end{enumerate}
Each one of us will take care of one module. The ``Extra'' module will contain content that complements the course. Some of this extra content includes a minigame on Quantum Key Distribution. Other extra content may be subject to changes due to the time limit.
\subsection{Exercises}
Since this course is focused on the hardware more than the software, we decided that quizzes would be better than notebook exercises. We are open to the idea of creating notebook exercises if time allows us, but it's not our first priority.
\subsection{Presentation}
We decided to keep the idea of the website, that can be reached \href{https://cosmcif.github.io/photonics-qworld-challenge/index.html}{\underline{here}}.\\As far as the hackathon goes, the website's only purpose will be to collect the course materials. After the hackathon ends, regardless of victory or loss, the website can be kept active for people to use to learn about Photonics using our modules.

\newpage
\section{Midway milestone}
What has our group achieved so far?
\begin{enumerate}
  \item The website is ready and working, though it's still full of dummy text.
  \item A minigame on Quantum Key Distribution has been developed and published.
  \item The first draft of the core modules is ready. 
  \item Some professionals have been contacted in order to be interviewed.
\end{enumerate}
Our goals for the second half of the challenges are:
\begin{enumerate}
  \item Polishing the modules and creating the exercises.
  \item Adding the modules to the website.
  \item Interviewing professionals in the field and publishing their answers.
\end{enumerate}
\section{After the hackathon}
The time constraint limits what we can create. Therefore, we believe this project shouldn't finish with the hackathon. We are committed towards mantaining the website and creating more content after the end of the hackathon. \\
Examples of additions that can be made after the end of the hackathon:
\begin{enumerate}
  \item In-depth modules about applications of photonics.
  \item In-depth modules about hardware and materials.
  \item More interactive content such as videogames or video tutorials.
  \item Contributions of experts in the field.
  \item Updated modules about state-of-the-art research and engineering developments.
  \item The creation of a community.
\end{enumerate}
\end{document}
